% DONE

\begin{frame}
\frametitle{Síntese}
\framesubtitle{Desidroalogenação de derivados monoalogenados}

Desidroalogenação com \ch{KOH} em \textbf{meio alcoólico}.

\begin{figure}
\centering
\scalebox{0.65}{
\schemestart
\chemfig{CH_3-[:0]CH(-[:270]CH_3)-[:0]CH(-[:90]Cl)-[:0]CH_3}
\+
\ch{KOH}
\arrow{->[*{0}\ch{ROH}]}
\chemfig{CH_3-[:0]CH(-[:270]CH_3)=[:0]CH-[:0]CH_3}
\+
\ch{KCl}
\+
\ch{H2O}
\schemestop
}
\end{figure}

Caso seja feita em meio aquoso, não gera alceno, e sim álcool.

\begin{figure}
\centering
\scalebox{0.65}{
\schemestart
\chemfig{CH_3-[:0]CH(-[:270]CH_3)-[:0]CH(-[:90]Cl)-[:0]CH_3}
\+
\ch{KOH}
\arrow{->[*{0}aquoso]}
\chemfig{CH_3-[:0]CH(-[:270]CH_3)-[:0]CH(-[:90] {\color{red} OH})-[:0]CH_3}
\schemestop
}
\end{figure}

\end{frame}
