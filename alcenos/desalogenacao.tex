% DONE

\begin{frame}
\frametitle{Síntese}
\framesubtitle{Desalogenação de derivado dialogenado}

Desalogenação a partir da reação com zinco em pó.

\begin{itemize}

\item \textbf{Derivado vicinal:} alceno com manutenção da cadeia
\begin{figure}
\centering
\scalebox{0.55}{
\schemestart
\chemfig{CH_3-[:0]CH_2-[:0]CH(-[:90]Cl)-[:0]CH_2(-[:90]Cl)}
\+
\ch{Zn_{(pó)}}
\arrow{->}
\chemfig{CH_3-[:0]CH_2-[:0]CH=[:0]CH_2}
\+
\ch{ZnCl2}
\schemestop
}
\end{figure}

\item \textbf{Derivado geminado:} alceno com duplicação da cadeia
\begin{figure}
\centering
\scalebox{0.55}{
\schemestart
\chemfig{2\, CH_3CH_2CH_2-[:0]CH(-[:270]Cl)-[:0]Cl}
\+
\ch{2 Zn_{(pó)}}
\arrow{->}
\chemfig{CH_3CH_2CH_2CH=[:0]CHCH_2CH_2CH_3}
\+
\ch{2 ZnCl2}
\schemestop
}
\end{figure}

\end{itemize}

\end{frame}

\begin{frame}
\frametitle{Síntese}
\framesubtitle{Desalogenação de derivado dialogenado}

Para derivados isolados, gera um composto cíclico.

\begin{figure}
\centering
\scalebox{0.65}{
\schemestart
\chemfig{CH_3-[:0]CH(-[:270]Cl)-[:0]CH_2-[:0]CH_2-[:270]Cl}
\+
\ch{Zn_{(pó)}}
\arrow{->}
\chemfig{*3(--(-[:180]CH_3)-)}
\+
\ch{ZnCl2}
\schemestop
}
\end{figure}

\end{frame}
