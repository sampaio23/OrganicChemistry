\begin{frame}
\frametitle{Síntese}
\framesubtitle{Desalogenação de derivado dialogenado}

Desalogenação a partir da reação com zinco em pó (\ch{Zn_{(pó)}}).

\begin{itemize}

\item \textbf{Derivado vicinal:} alceno com manutenção da cadeia
\begin{figure}
\centering
\scalebox{0.4}{
\schemestart
\chemfig{CH_3-[:0]CH_2-[:0]CH(-[:90]Cl)-[:0]CH2(-[:90]Cl)}
\+
\ch{Zn_{(pó)}}
\arrow{->}
\chemfig{CH_3-[:0]CH_2-[:0]CH=[:0]CH2}
\+
\ch{ZnCl2}
\schemestop
}
\end{figure}

\item \textbf{Derivado geminado:} alceno com duplicação da cadeia
\begin{figure}
\centering
\scalebox{0.1}{
\schemestart
\chemfig{R(-[:0]CH(=[:0]CH_2))}
\arrow{->[*{0}\color{blue}\ch{H2O}][*{0}\ch{H+}]}
\chemfig{R(-[:0]CH(-[:270] {\color{blue}OH})(-[:0]CH_2(-[:270] {\color{blue}H})))}
\schemestop
}
\end{figure}

\item \textbf{Derivado isolado:} composto cíclico
\begin{figure}
\centering
\scalebox{0.1}{
\schemestart
\chemfig{R(-[:0]CH(=[:0]CH_2))}
\arrow{->[*{0}\color{blue}\ch{H2O}][*{0}\ch{H+}]}
\chemfig{R(-[:0]CH(-[:270] {\color{blue}OH})(-[:0]CH_2(-[:270] {\color{blue}H})))}
\schemestop
}
\end{figure}

\end{itemize}

\end{frame}
