% DONE

\begin{frame}
\frametitle{Síntese}
\framesubtitle{Desidratação de Álcoois}

Desidratação intramolecular de álcoois por ácido sulfúrico.

\begin{figure}
\centering
\scalebox{1}{
\schemestart
\chemfig{CH_2(-[:0]@{R}CH_2(-[:270] {\color{blue}OH}))(-[:270] {\color{blue}H})}
\arrow{->[*{0}\ch{H2SO4}][*{0}$170^{\circ}C$]}
\chemfig{CH_2(=[:0]CH_2)}
\+
\ch{\color{blue}H2O}
% \chemmove{
%   \draw[
%     fill=purple,
%     draw=purple,
%     fill opacity=.2,
%     rounded corners=2pt
%   ]
%     ([xshift=-10pt,yshift=-3pt]R.south west)
%     rectangle
%     ([xshift=4pt,yshift=3pt]R.north east)   node[xshift=0pt,yshift=5pt,above,opacity=1]{Mais estável};
% }
\schemestop
}
\end{figure}

A temperatura necessária para desidratação depende da classificação do álcool:
\begin{itemize}
\item Álcool primário: $\sim$ $170^{\circ}C$
\item Álcool secundário: $\sim$ $90^{\circ}C$
\item Álcool terciário: $\sim$ $30^{\circ}C$
\end{itemize}

\end{frame}

\begin{frame}
\frametitle{Síntese}
\framesubtitle{Desidratação de Álcoois}

\textbf{Regra de Saytzeff:} estuda a estabilidade dos alcenos por meio dos calores de hidrogenação. Diz que:\\
\textit{Quanto mais substituído, mais estável é o alceno.}

\begin{figure}
\centering
\scalebox{0.45}{
\schemestart
\chemfig{-[:30](-[:90]H)-[:-30](-[:270]OH)-[:30]}
\arrow{->[*{0}\ch{H2SO4}][*{0}$\Delta$]}
\chemfig{-[:30]-[:-30]=[:30]}
\+
\chemfig{-[:30]@{R}=[:-30]-[:30]}
\+
\ch{H2O}
\chemmove{
  \draw[
    fill=purple,
    draw=red,
    fill opacity=.2,
    rounded corners=2pt
  ]
    ([xshift=60pt,yshift=20pt]R.south east)
    rectangle
    ([xshift=-30pt,yshift=-30pt]R.north west)   node[xshift=45pt,yshift=-10pt,below,opacity=1]{Mais estável};
}
\schemestop
}
\end{figure}

\begin{figure}
\centering
\scalebox{0.45}{
\schemestart
\chemfig{*5(---([:120]-)(-[:60]OH)--)}
\arrow{->[*{0}\ch{H2SO4}][*{0}$\Delta$]}
\chemfig{*5(---([:90]=)--)}
\+
\chemfig{*5(-@{R}-=([:90]-)--)}
\+
\ch{H2O}
\chemmove{
  \draw[
    fill=purple,
    draw=red,
    fill opacity=.18,
    rounded corners=2pt
  ]
    ([xshift=23pt,yshift=90pt]R.south east)
    rectangle
    ([xshift=-35pt,yshift=-10pt]R.north west)   node[xshift=30pt,yshift=-8pt,below,opacity=1]{Mais estável};
}
\schemestop
}
\end{figure}

\end{frame}
