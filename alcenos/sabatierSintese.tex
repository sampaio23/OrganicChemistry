% DONE

\begin{frame}
\frametitle{Síntese}
\framesubtitle{Hidrogenação Parcial de Alcinos}

Adição de \ch{H2} a alcinos na presença de catalisador e sulfato de bário como \textit{veneno}.

\begin{figure}
\centering
\scalebox{0.95}{
\schemestart
\chemfig{R(-[:0]C(~[:0]C(-[:0]R')))}
\+
\ch{H2}
\arrow{->[*{0}\ch{Pd}][*{0}\ch{BaSO4}]}
\chemfig{R(-[:0]C(-[:270]H)(=[:0]C(-[:270]H)(-[:0]R')))}
\schemestop
}
\end{figure}

O \ch{BaSO4} envenena a solução, pois gruda no catalisador de forma a desacelerar a segunda hidrogenação, impedindo a passagem de
alceno para alcano.\\
É uma reação estereoespecífica, pois o alceno formado é o \textbf{cis} (nos casos de isomeria geométrica).

\end{frame}
