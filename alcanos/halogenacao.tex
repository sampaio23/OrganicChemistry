\begin{frame}
\frametitle{Reações}
\framesubtitle{Halogenação}

Reação entre alcanos e halogênios.

\begin{figure}
\centering
\scalebox{.8}{
\schemestart
\chemfig{C(-[:0]H)(-[:90]H)(-[:180]H)(-[:270] {\color{red}H})} 
\+
\ch{\color{blue}Cl2} 
\arrow{->[*{0}Luz][*{0}$\Delta$]} 
\chemfig{C(-[:0]H)(-[:90]H)(-[:180]H)(-[:270] {\color{blue}Cl})} 
\+
\ch{\color{red}H{\color{blue}Cl}}
\schemestop
}
\end{figure}

A reação pode continuar:

\begin{figure}
\centering
\scalebox{.5}{
\schemestart
\chemfig{C(-[:0]H)(-[:90]H)(-[:180]H)(-[:270]Cl)}  
\arrow{->[*{0}\ch{Cl2}][*{0}$\Delta$]} 
\chemfig{C(-[:0]H)(-[:90]H)(-[:180]Cl)(-[:270]Cl)} 
\+
\ch{HCl}
\arrow{->[*{0}\ch{Cl2}][*{0}$\Delta$]} 
\chemfig{C(-[:0]H)(-[:90]Cl)(-[:180]Cl)(-[:270]Cl)} 
\+
\ch{2 HCl}
\arrow{->[*{0}\ch{Cl2}][*{0}$\Delta$]} 
\chemfig{C(-[:0]Cl)(-[:90]Cl)(-[:180]Cl)(-[:270]Cl)} 
\+
\ch{3 HCl}
\schemestop
}
\end{figure}

\end{frame}



\begin{frame}
\frametitle{Reações}
\framesubtitle{Halogenação}

Deve-se levar em consideração a reatividade de hidrogênios e de halogênios.

\vspace{0.7cm}

\begin{minipage}{1\linewidth}
  \centering
  \begin{tabular}{cc}
    \begin{tabular}{c}
      \textbf{Hidrogênio a carbono:}\\
      Terciário $>$ Secundário $>$ Primário
    \end{tabular}
    &
    \begin{tabular}{c}
      \textbf{Halogênios:}\\
      \ch{F2} $>$ \ch{Cl2} $>$ \ch{Br2} $>$ \ch{I2}
    \end{tabular}
  \end{tabular}
\end{minipage}

\vspace{0.7cm}

Observe o exemplo abaixo:

\begin{figure}
\centering
\scalebox{.8}{
\schemestart
\chemfig{-[:30]([:-30]-)([:90]-)}
\arrow{->[*{0}\ch{Br2}][*{0}$Luz$]}
\chemname{\chemfig{-[:30]([:-30]-)([:90]-)(-[:270]Br)}}{93\%}
\+
\chemname{\chemfig{-[:30](-[:-30]-[:30]Br)([:90]-)}}{7\%}
\schemestop
}
\end{figure}

\end{frame}
